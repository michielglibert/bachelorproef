%---------- Inleiding ---------------------------------------------------------

\section{Introductie} % The \section*{} command stops section numbering
\label{sec:introductie}

Virtual realitiy is een opkomende trend maar staat nog steeds in zijn kinderschoenen. Er is dus nood aan applicaties die gebruik maken van deze technologie. Voor een developer is dit een geheel nieuwe manier van ontwikkelen. React 360 zorgt ervoor dat er aan de hand van de huidige technologie (React JS) een virtual reality webapplicatie kan worden gemaakt. Zo kan de developer dus vooral focussen op de principes die er in de virtual reality wereld heersen en minder op het technische gedeelte. Ik ga onderzoeken of React 360 een geschikt framework is voor gebruiksvriendelijke applicaties te ontwikkelen. Bij virtual reality moet er met een aantal andere aspecten worden rekening gehouden. Deze applicaties moeten niet alleen een goede user experience bezitten maar ook de gebruiker minimale fysieke last bezorgen. Hier heb ik het dus over duiziligheid, misselijkheid, enz.. Ik ga daarnaast ook bekijken voor welke doeleinden het framework gebruikt kan worden. Bijvoorbeeld voor gaming, entertainment, medische sector, ...

%---------- Stand van zaken ---------------------------------------------------

\section{State-of-the-art}
\label{sec:state-of-the-art}

React 360 is nog een vrije nieuwe technologie die nog maar sinds 2017 bestaat. Het is dus ook een framework dat zoals virtual reality zelf nog volop in ontwikkeling is. De API die gebruikt wordt in de browsers om virtual reality via webpagina's te ondersteunen noemt WebVR. Dit zal dus hoogstwaarschijnlijk enkele malen aanbod komen in mijn onderzoek. De website van React zelf is goed gedocumenteerd en zal dus ook zeker een startpunt vormen voor het verzamelen van informatie. Daarnaast zijn ook enkele boeken ter beschikking waarbij ik die van Michael Mangialardi, Learn React 360 \autocite{Mangialardi2017} ook zal gebruiken voor React 360 aan te leren.

%---------- Methodologie ------------------------------------------------------
\section{Methodologie}
\label{sec:methodologie}

Ik zal mijn onderzoek beginnen door eerst te bekijken wat React 360 allemaal te bieden heeft. Ik ga dus zelf ook met React 360 leren werken en hiervoor enkele applicaties gaan schrijven. Ik zal deze applicaties testen in zowel 360\textdegree \hspace{0em} als met een VR bril voor mijn smartphone. Deze apps zullen niet alleen door mij getest worden maar ook door een aantal andere mensen waar ik naar hun meningen zal vragen over enkele onderwerpen die aan virtual reality gebonden zijn. Met deze informatie zal ik gaan kijken wat de mogelijkheden allemaal zijn van het framework en of het dus wel geschikt is in zijn huidige vorm om gebruiksvriendelijke virtual reality applicaties te maken. Verder zal ik dan hieruit gaan kijken voor welke doeleinden het dan geschikt is. 

%---------- Verwachte resultaten ----------------------------------------------
\section{Verwachte resultaten}
\label{sec:verwachte_resultaten}

Ik verwacht dat React 360 voor zowel de developer als voor de gebruiker een geschikt framework is waarmee gebruiksvriendelijke applicaties ontwikkeld kunnen worden die bestemdt zijn voor een groot aantal doeleinden. Daarnaast verwacht ik dat de applicaties die ik zal schrijven goed zullen ontvangen worden bij de mensen die ik het laat testen. Maar men zal virtual reality nog eerder zien als een 'gimmick' die niet echt nodig is. Ik verwacht uiteindelijk ook dat er enkele fysieke lasten zullen optreden.
%---------- Verwachte conclusies ----------------------------------------------
\section{Verwachte conclusies}
\label{sec:verwachte_conclusies}

Ik verwacht dat React 360 een geschikt framework is voor gebruiksvriendelijke applicaties te ontwikkelen die gebruikt kan worden voor een groot aantal doeleinden. Alhoewel ik niet verwacht dat het framework zelf zeer uitgebreid zal zijn, verwacht ik wel dat het een zeer goede basis is waar zeker nog in de toekomst verder op kan gebouwd worden.
