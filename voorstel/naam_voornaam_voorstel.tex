%==============================================================================
% Sjabloon onderzoeksvoorstel bachelorproef
%==============================================================================
% Gebaseerd op LaTeX-sjabloon ‘Stylish Article’ (zie voorstel.cls)
% Auteur: Jens Buysse, Bert Van Vreckem

\documentclass[fleqn,10pt]{voorstel}

%------------------------------------------------------------------------------
% Metadata over het voorstel
%------------------------------------------------------------------------------

\JournalInfo{HoGent Bedrijf en Organisatie}
\Archive{Bachelorproef 2017 - 2018} % Of: Onderzoekstechnieken

%---------- Titel & auteur ----------------------------------------------------

% TODO: geef werktitel van je eigen voorstel op
\PaperTitle{React VR, een framework voor virtual reality applicaties te bouwen}
\PaperType{Onderzoeksvoorstel Bachelorproef} % Type document

% TODO: vul je eigen naam in als auteur, geef ook je emailadres mee!
\Authors{Michiel Glibert\textsuperscript{1}} % Authors
\CoPromotor{Jasper Dansercoer\textsuperscript{2} (Kayzr)}
\affiliation{\textbf{Contact:}
  \textsuperscript{1} \href{mailto:michiel.glibert.w2493@student.hogent.be}{michiel.glibert.w2493@student.hogent.be};
  \textsuperscript{2} \href{mailto:jasper@kayzr.com}{jasper@kayzr.com};
}

%---------- Abstract ----------------------------------------------------------

\Abstract{Virtual reality is een opkomende trend die nog steeds in zijn kinderschoenen staat, maar sneller en sneller aan het groeien is. De technologie wordt steeds beter en uitgebreider. Daardoor is er ook een nood aan meer applicaties die gebruik maken van deze technologie. React VR is een framework waarmee er gemakkelijk aan de hand van de javascript library, ReactJS, virtual reality applicaties kunnen geschreven worden. In mijn bachelorproef zal ik onderzoeken of React VR een geschikt framework is voor het ontwikkelen van gebruiksvriendelijke virtual reality applicaties. Daarnaast zal ik bekijken wat de mogelijkheden van dit framework zijn en voor welke doeleinden het kan gebruikt worden. Ik verwacht dan ook dat React VR een geschikt framework zal zijn dat voor een groot aantal doeleinden kan gebruikt worden.
}

%---------- Onderzoeksdomein en sleutelwoorden --------------------------------
% TODO: Sleutelwoorden:
%
% Het eerste sleutelwoord beschrijft het onderzoeksdomein. Je kan kiezen uit
% deze lijst:
%
% - Mobiele applicatieontwikkeling
% - Webapplicatieontwikkeling
% - Applicatieontwikkeling (andere)
% - Systeembeheer
% - Netwerkbeheer
% - Mainframe
% - E-business
% - Databanken en big data
% - Machineleertechnieken en kunstmatige intelligentie
% - Andere (specifieer)
%
% De andere sleutelwoorden zijn vrij te kiezen

\Keywords{Onderzoeksdomein. Webapplicatieontwikkeling --- Virtual Reality --- Javascript} % Keywords
\newcommand{\keywordname}{Sleutelwoorden} % Defines the keywords heading name

%---------- Titel, inhoud -----------------------------------------------------

\begin{document}

\flushbottom % Makes all text pages the same height
\maketitle % Print the title and abstract box
\tableofcontents % Print the contents section
\thispagestyle{empty} % Removes page numbering from the first page

%------------------------------------------------------------------------------
% Hoofdtekst
%------------------------------------------------------------------------------

% De hoofdtekst van het voorstel zit in een apart bestand, zodat het makkelijk
% kan opgenomen worden in de bijlagen van de bachelorproef zelf.
%---------- Inleiding ---------------------------------------------------------

\section{Introductie} % The \section*{} command stops section numbering
\label{sec:introductie}

Virtual reality is een opkomende trend maar staat nog steeds in zijn kinderschoenen. Er is dus nood aan applicaties die gebruik maken van deze technologie. Voor een ontwikkelaar is dit een geheel nieuwe manier van ontwikkelen. React 360 zorgt ervoor dat er aan de hand van de huidige technologie (React JS) een virtual reality webapplicatie kan worden gemaakt. Zo kan de ontwikkelaar dus vooral focussen op de principes die er in de virtual reality wereld heersen en minder op het technische gedeelte. Ik ga onderzoeken of React 360 een geschikt framework is voor gebruiksvriendelijke applicaties te ontwikkelen. Bij virtual reality moet er met een aantal aspecten rekening worden gehouden. Deze applicaties moeten niet alleen een goede user experience bezitten maar ook de gebruiker minimale fysieke last bezorgen. Hier heb ik het dus over duiziligheid, misselijkheid... Ik ga daarnaast ook bekijken voor welke doeleinden het framework gebruikt kan worden. Bijvoorbeeld voor gaming, entertainment, medische sector...

%---------- Stand van zaken ---------------------------------------------------

\section{State-of-the-art}
\label{sec:state-of-the-art}

React 360 is nog een vrije nieuwe technologie die nog maar sinds 2017 bestaat. Het is een framework dat net zoals virtual reality zelf, nog volop in ontwikkeling is. De API die gebruikt wordt in de browsers om virtual reality via webpagina's te ondersteunen heet WebVR. De website van React zelf is goed gedocumenteerd en zal dus ook zeker een startpunt vormen voor het verzamelen van informatie. Daarnaast zijn ook enkele boeken ter beschikking, waaronder één van Michael Mangialardi, Learn React 360 (Mangialardi, 2017), die ik ook zal gebruiken zodat ik React 360 leer gebruiken.

%---------- Methodologie ------------------------------------------------------
\section{Methodologie}
\label{sec:methodologie}

Ik zal mijn onderzoek beginnen door eerst te bekijken wat React 360 allemaal te bieden heeft. Ik zal zelf ook met React 360 leren werken en hiervoor enkele applicaties gaan schrijven. Ik zal deze applicaties testen in zowel 360\textdegree \hspace{0em} als met een VR bril voor mijn smartphone. Deze apps zullen niet alleen door mij getest worden maar ook door een aantal andere mensen waarna ik naar hun meningen zal vragen over enkele onderwerpen die aan virtual reality gebonden zijn. Met deze informatie zal ik gaan kijken wat de mogelijkheden allemaal zijn van het framework en of het dus wel geschikt is in zijn huidige vorm om gebruiksvriendelijke virtual reality applicaties te maken. Hieruit zal ik gaan kijken voor welke doeleinden het framework geschikt is.

%---------- Verwachte resultaten ----------------------------------------------
\section{Verwachte resultaten}
\label{sec:verwachte_resultaten}

Ik verwacht dat React 360 voor zowel de ontwikkelaar als voor de gebruiker een geschikt framework is waarmee gebruiksvriendelijke applicaties ontwikkeld kunnen worden die bestemd zijn voor een groot aantal doeleinden. Daarnaast verwacht ik dat de applicaties die ik zal schrijven goed zullen ontvangen wordenbij de mensen bij wie ik die laat testen. Maar men zal virtual reality nog eerder zien als een 'gimmick' die niet echt nodig is. Ik verwacht uiteindelijk ook dat er enkele fysieke lasten zullen optreden.
%---------- Verwachte conclusies ----------------------------------------------
\section{Verwachte conclusies}
\label{sec:verwachte_conclusies}

Ik denk dat uit mijn onderzoek zal blijken React 360 een geschikt framework is voor gebruiksvriendelijke applicaties te ontwikkelen die gebruikt kunnen worden voor een groot aantal doeleinden. Alhoewel ik niet verwacht dat het framework zelf zeer uitgebreid zal zijn, verwacht ik wel dat het een zeer goede basis is waar zeker nog in de toekomst verder op kan worden gebouwd.


%------------------------------------------------------------------------------
% Referentielijst
%------------------------------------------------------------------------------
% TODO: de gerefereerde werken moeten in BibTeX-bestand ``voorstel.bib''
% voorkomen. Gebruik JabRef om je bibliografie bij te houden en vergeet niet
% om compatibiliteit met Biber/BibLaTeX aan te zetten (File > Switch to
% BibLaTeX mode)

\phantomsection
\printbibliography[heading=bibintoc]

\end{document}
