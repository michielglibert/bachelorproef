%==============================================================================
% Sjabloon onderzoeksvoorstel bachelorproef
%==============================================================================
% Gebaseerd op LaTeX-sjabloon ‘Stylish Article’ (zie voorstel.cls)
% Auteur: Jens Buysse, Bert Van Vreckem

% TODO: Compileren document:
% 1) Vervang ‘naam_voornaam’ in de bestandsnaam door je eigen naam, bv.
%    buysse_jens_voorstel.tex
% 2) latexmk -pdf naam_voornaam_voorstel.tex
% 3) biber naam_voornaam_voorstel
% 4) latexmk -pdf naam_voornaam_voorstel.tex (1 keer)

\documentclass[fleqn,10pt]{voorstel}
\usepackage{textcomp}

%------------------------------------------------------------------------------
% Metadata over het artikel
%------------------------------------------------------------------------------

\JournalInfo{HoGent Bedrijf en Organisatie} % Journal information
\Archive{Onderzoekstechnieken 2016 - 2017} % Additional notes (e.g. copyright, DOI, review/research article)

%---------- Titel & auteur ----------------------------------------------------

% TODO: geef werktitel van je eigen voorstel op
\PaperTitle{React VR, een framework voor virtual reality applicaties te bouwen}
\PaperType{Onderzoeksvoorstel Bachelorproef} % Type document

% TODO: vul je eigen naam in als auteur, geef ook je emailadres mee!
\Authors{Michiel Glibert} % Authors
\affiliation{\textbf{Contact:}
  \href{mailto:michiel.glibert.w2493@student.hogent.be}
  {michiel.glibert.w2493@student.hogent.be}}

%---------- Abstract ----------------------------------------------------------

  \Abstract{
  	Virtual reality is een opkomende trend die nog steeds in zijn kinderschoenen staat, maar sneller en sneller aan het groeien is. De technologie wordt steeds beter en uitgebreider. Daardoor is er ook een nood aan meer applicaties die virtual reality ondersteunen. React VR is een framework waarmee er gemakkelijk aan de hand van de javascript library, ReactJS, virtual reality applicaties kunnen geschreven worden. In mijn bachelorproef zal ik onderzoeken of React VR een geschikt framework is voor het ontwikkelen van gebruiksvriendelijke virtual reality applicaties. Daarnaast zal ik bekijken wat de mogelijkheden van dit framework zijn en voor welke doeleinden het kan gebruikt worden. Ik verwacht dan ook dat React VR een geschikt framework zal zijn dat voor een zeer aantal doeleinden kan gebruikt worden.}

%---------- Onderzoeksdomein en sleutelwoorden --------------------------------
% TODO: Sleutelwoorden:
%
% Het eerste sleutelwoord beschrijft het onderzoeksdomein. Je kan kiezen uit
% deze lijst:
%
% - Mobiele applicatieontwikkeling
% - Webapplicatieontwikkeling
% - Applicatieontwikkeling (andere)
% - Systeem- en netwerkbeheer
% - Mainframe
% - E-business
% - Databanken en big data
% - Machine learning en kunstmatige intelligentie
% - Andere (specifieer)
%
% De andere sleutelwoorden zijn vrij te kiezen

\Keywords{Onderzoeksdomein. Webapplicatieontwikkeling --- Virtual Reality --- Javascript} %Keywords
\newcommand{\keywordname}{Sleutelwoorden} % Defines the keywords heading name

%---------- Titel, inhoud -----------------------------------------------------
\begin{document}

\flushbottom % Makes all text pages the same height
\maketitle % Print the title and abstract box
\tableofcontents % Print the contents section
\thispagestyle{empty} % Removes page numbering from the first page

%------------------------------------------------------------------------------
% Hoofdtekst
%------------------------------------------------------------------------------

%---------- Inleiding ---------------------------------------------------------

\section{Introductie} % The \section*{} command stops section numbering
\label{sec:introductie}

Virtual realitiy is een opkomende trend maar staat nog steeds in zijn kinderschoenen. Er is dus nood aan applicaties die virtual reality ondersteunen. Voor een developer is dit een geheel nieuwe manier van ontwikkelen. React VR zorgt ervoor dat er aan de hand van de huidige technologie (ReactJS) een virtual reality webapplicatie kan worden gemaakt. Zo kan de developer dus vooral focussen op de principes die er in de virtual reality wereld heersen en minder op het technische gedeelte. Ik ga onderzoeken of React VR een geschikt framework is voor gebruiksvriendelijke applicaties te ontwikkelen. Bij virtual reality moet er met een aantal andere aspecten worden rekening gehouden. Deze applicaties moeten niet alleen een goede user experience bezitten maar ook de gebruiker minimale fysieke last bezorgen. Hier heb ik het dus over duiziligheid, misselijkheid, enz.. Hier ga ik niet dieper op in in mijn onderzoek. Ik ga daarnaast ook bekijken voor welke doeleinden het framework gebruikt kan worden. Bijvoorbeeld voor gaming, entertainment, medische sector, enz.

%---------- Stand van zaken ---------------------------------------------------

\section{Stand van zaken}
\label{sec:state-of-the-art}

React VR is nog een vrije nieuwe technologie die nog maar sinds 2017 bestaat. Het is dus ook een framework dat zoals virtual reality zelf nog volop in ontwikkeling is. De API die gebruikt wordt in de browsers om virtual reality via webpagina's te ondersteunen noemt WebVR. Dit zal dus hoogstwaarschijnlijk enkele malen aanbod komen in mijn onderzoek. De website van React zelf is goed gedocumenteerd en zal dus ook zeker een startpunt vormen voor het verzamelen van informatie. Daarnaast zijn ook enkele boeken ter beschikking waarbij ik die van Michael Mangialardi, Learn React VR \autocite{Mangialardi2017} ook zal gebruiken voor React VR aan te leren.

% Voor literatuurverwijzingen zijn er twee belangrijke commando's:
% \autocite{KEY} => (Auteur, jaartal) Gebruik dit als de naam van de auteur
%   geen onderdeel is van de zin.
% \textcite{KEY} => Auteur (jaartal)  Gebruik dit als de auteursnaam wel een
%   functie heeft in de zin (bv. ``Uit onderzoek door Doll & Hill (1954) bleek
%   ...'')

%---------- Methodologie ------------------------------------------------------
\section{Methodologie}
\label{sec:methodologie}

Ik zal mijn onderzoek beginnen door eerst te bekijken wat React VR allemaal te bieden heeft. Ik ga dus zelf ook met React VR leren werken en hiervoor enkele applicaties gaan schrijven. Ik zal deze applicaties testen in zowel 360\textdegree \hspace{0em} als met een VR bril voor mijn smartphone. Deze apps zullen niet alleen door mij getest worden maar ook door een aantal andere mensen waar ik naar hun meningen zal vragen over enkele onderwerpen die aan virtual reality gebonden zijn. Met deze informatie zal ik gaan kijken wat de mogelijkheden allemaal zijn van het framework en of het dus wel geschikt is in zijn huidige vorm om gebruiksvriendelijke virtual reality applicaties te maken. Verder zal ik dan hieruit gaan kijken voor welke doeleinden het dan geschikt is. 
%---------- Verwachte resultaten ----------------------------------------------
\section{Verwachte resultaten}
\label{sec:verwachte_resultaten}

Ik verwacht dat React VR voor zowel de developer als voor de gebruiker een geschikt framework is waarmee gebruiksvriendelijke applicaties ontwikkeld kunnen worden die bestemdt zijn voor een groot aantal doeleinden. Daarnaast verwacht ik dat de applicaties die ik zal schrijven goed zullen ontvangen worden bij de mensen die ik het laat testen. Maar men zal virtual reality nog eerder zien als een 'gimmick' die niet echt nodig is. Ik verwacht uiteindelijk ook dat er enkele fysieke lasten zullen optreden. Hiermee heb ik het over zweten, duiziligheid, misselijkheid, ... maar hier ga ik niet dieper op in in mijn onderzoek.

%---------- Verwachte conclusies ----------------------------------------------
\section{Verwachte conclusies}
\label{sec:verwachte_conclusies}

Ik verwacht dat React VR een geschikt framework is voor gebruiksvriendelijke applicaties te ontwikkelen die gebruikt kan worden voor een groot aantal doeleinden. Alhoewel ik niet verwacht dat het framework zelf zeer uitgebreid zal zijn, verwacht ik wel dat het een zeer goede basis is waar zeker nog in de toekomst verder op kan gebouwd worden.
%------------------------------------------------------------------------------
% Referentielijst
%------------------------------------------------------------------------------
% TODO: de gerefereerde werken moeten in BibTeX-bestand ``biblio.bib''
% voorkomen. Gebruik JabRef om je bibliografie bij te houden en vergeet niet
% om compatibiliteit met Biber/BibLaTeX aan te zetten (File > Switch to
% BibLaTeX mode)

\phantomsection
\printbibliography[heading=bibintoc]

\end{document}
