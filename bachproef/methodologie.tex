%%=============================================================================
%% Methodologie
%%=============================================================================

\chapter{Methodologie}
\label{ch:methodologie}

%% TODO: Hoe ben je te werk gegaan? Verdeel je onderzoek in grote fasen, en
%% licht in elke fase toe welke stappen je gevolgd hebt. Verantwoord waarom je
%% op deze manier te werk gegaan bent. Je moet kunnen aantonen dat je de best
%% mogelijke manier toegepast hebt om een antwoord te vinden op de
%% onderzoeksvraag.

\section{Het framework, React 360, bestuderen}
\label{sec:reactvr-bestuderen}
We gaan eerst kort React.JS zelf bekijken. Hiermee wordt React 360 namelijk opgebouwd en het is dus cruciaal om hier ook kennis over te hebben. Daarna zullen we de belangrijkste concepten van het React 360 framework gaan bekijken. Zowel de manier van opbouw in de applicatie, als de extra beschikbare componenten in het framework.

\section{Ervaring van virtual reality voor een gebruiker}
\label{sec:ervaring-vr-app}
Daarna kunnen we met al de informatie die we verworven hebben door middel van React 360 enkele eenvoudige applicaties gaan maken. Deze applicaties zullen getest worden door een aantal personen en zij zullen dan hun mening geven a.d.h.v. een korte vragenlijst. Deze zal dan een beter inzicht geven in hoe een persoon een virtual reality applicatie gemaakt met React 360 ervaart.

\subsubsection{App 1}
De eerste applicatie zal de simpelste zijn. Dit is een applicatie waarbij de nadruk ligt op het simuleren van een echte omgeving. Hiervoor zal een gebied dat echt bestaat in 360\textdegree worden weergegeven met het bijhorende omgevingsgeluid. De gebruiker zal dan kunnen ervaren alsof hij echt op die locatie aanwezig is.

\subsubsection{App 2}
De 2de applicatie zal meer de nadruk leggen op de interactie met de virtuele wereld. Er zal hier dus vooral gekeken worden naar de input mogelijkheden en hoe de gebruiker dit ervaart. Zo zal er een applicatie worden gebouwd met verscheidende knoppen waarop de gebruiker kan drukken a.d.h.v. een knop de virtuele headset. Er zullen dan bepaalde elementen in de gebruikersomgeving zich gaan aanpassen afhankelijk van wat de gebruiker geselecteerd heeft.

\subsubsection{Prestaties}
We gaan ook tijdens testen van deze applicaties letten op hoe de applicaties presteren. Hoeveel geheugen verbruiken ze? Hoeveel aanslag leggen ze op de batterij? Blokkeert de browser soms?

\section{Conclusie: De doeleinden en stand van zaken van React 360}
\label{sec:doeleinden-reactvr}
Ten slotte zullen we een besluit opmaken van React 360. Hier gaan we dan een stand van zaken opstellen hoe React 360 scoort als framework voor virtual reality. Zo zullen we antwoord formuleren op alle onderzoeksvragen.



