%%=============================================================================
%% Methodologie
%%=============================================================================

\chapter{Methodologie}
\label{ch:methodologie}

%% TODO: Hoe ben je te werk gegaan? Verdeel je onderzoek in grote fasen, en
%% licht in elke fase toe welke stappen je gevolgd hebt. Verantwoord waarom je
%% op deze manier te werk gegaan bent. Je moet kunnen aantonen dat je de best
%% mogelijke manier toegepast hebt om een antwoord te vinden op de
%% onderzoeksvraag.

\section{Dieper op de concepten van VR ingaan}
\label{sec:dieper-op-vr-ingaan}
Eerst zullen we dieper op virtual reality zelf ingaan. We gaan bekijken hoe virtual reality precies werkt, hoe het gevolueerd is over de voorbije jaren en wat de mogelijke frameworks momenteel zijn voor het maken van een virtual reality applicatie. We bekijken ook de gevolgen voor de gebruiker van een virtual reality applicatie, dit is namelijk geen gewone webpagina meer. Ten slotte gaan we ook nog kort zien wat de toekomst van virtual reality is en of het wel degelijk the next big thing kan worden.

\section{Het framework, React VR, bestuderen}
\label{sec:reactvr-bestuderen}
Vervolgens gaan we over naar het React VR framework. We gaan eerst kort React.JS zelf bekijken. Hiermee wordt React VR namelijk opgebouwd en is dus cruciaal om hier ook kennis over te hebben. Daarna zullen we de belangrijkste delen van het React VR framework gaan bekijken. Zowel de manier van opbouw in de applicatie, als de beschikbare componenten in het framework. Ten slotte nemen we een kijkje naar hoe zo een React VR applicatie presteert.

\section{Onderzoeken hoe een gemiddelde persoon virtual reality ervaart}
\label{sec:ervaring-vr-app}
Daarna kunnen we met al de informatie die we verworven hebben doormiddel van React VR enkele eenvoudige applicaties gaan maken. Deze applicaties zullen getest worden door enkele personen en zij zullen dan hun mening geven a.d.h.v. een korte vragenlijst. Deze zal dan een beter inzicht geven in hoe een persoon een virtual reality applicatie ervaart.
App 1
De eerste applicatie zal de simpelste zijn. Dit is een applicatie waarbij de nadruk ligt op het simuleren van een echt omgeving. Hiervoor zal een gebied dat echt bestaat in 360\textdegree worden weergegeven met het bijhorende omgevingsgeluid. 
App 2
De 2de applicatie zal meer de nadruk leggen op de interactie met de virtuele wereld. Er zal hier dus vooral gekeken worden naar de input mogelijkheden en hoe de gebruiker dit ervaart.
App 3
…?

\section{De doeleinden en stand van zaken van React VR}
\label{sec:doeleinden-reactvr}
Ten slotte zullen we een besluit opmaken van React VR. We gaan bekijken voor welke doeleinden we React VR zouden kunnen gebruiken en hoe het scoort op vlak van performantie. Hier gaan we dan een stand van zaken opstellen hoe React VR scoort als framework voor virtual reality. We zullen dan ook een conclusie kunnen opmaken van dit onderzoek.



