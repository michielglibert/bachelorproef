%%=============================================================================
%% Methodologie
%%=============================================================================

\chapter{Methodologie}
\label{ch:methodologie}

%% TODO: Hoe ben je te werk gegaan? Verdeel je onderzoek in grote fasen, en
%% licht in elke fase toe welke stappen je gevolgd hebt. Verantwoord waarom je
%% op deze manier te werk gegaan bent. Je moet kunnen aantonen dat je de best
%% mogelijke manier toegepast hebt om een antwoord te vinden op de
%% onderzoeksvraag.

\section{Het framework, React 360, bestuderen}
\label{sec:reactvr-bestuderen}
We bekeken eerst kort React.JS zelf. Hiermee wordt React 360 namelijk opgebouwd en het was dus cruciaal om hier ook kennis over te hebben. Daarna bekeken we de belangrijkste concepten van het React 360 framework. Zowel de manier van opbouw in de applicatie, als de extra beschikbare componenten in het framework.

\section{Ervaring van virtual reality voor een gebruiker}
\label{sec:ervaring-vr-app}
Met al de informatie die we verworven hadden, konden we door middel van React 360 enkele eenvoudige applicaties maken. Deze applicaties werden getest door een aantal personen en zij hadden dan hun mening kunnen geven a.d.h.v. een korte vragenlijst. Deze gaf dan een beter inzicht in hoe een persoon een virtual reality applicatie gemaakt met React 360, ervaart.

\subsubsection{App 1}
De eerste applicatie was de meest eenvoudige. Dit is een applicatie waarbij de nadruk lag op het simuleren van een echte omgeving. Hiervoor werd een gebied dat echt bestaat in 360\textdegree\ weergegeven met het bijhorende omgevingsgeluid. De gebruiker kon dan ervaren alsof hij echt op die locatie aanwezig was.

\subsubsection{App 2}
De 2de applicatie legde meer de nadruk op de interactie met de virtuele wereld. Er werd hier dus vooral gekeken naar de input mogelijkheden en hoe de gebruiker dit ervaarde. Zo was de applicatie opgebouwd met verschillende knoppen waarop de gebruiker kon drukken a.d.h.v. een fysieke knop die aanwezig was op de virtuele headset die gebruikt werd. Er paste zich dan bepaalde elementen aan in de gebruikersomgeving afhankelijk van wat de gebruiker gekozen had.

\subsubsection{Prestaties}
Tijdens het testen van deze applicaties hielden we rekening met de prestaties van de applicaties. Hoeveel geheugen verbruikten ze? Hoeveel aanslag legden ze op de batterij? Blokkeerde de browser soms?

\section{Conclusie: De doeleinden en stand van zaken van React 360}
\label{sec:doeleinden-reactvr}
Ten slotte werd er dan een besluit opgemaakt van dit onderzoek. Hier werd een stand van zaken opgesteld over hoe React 360 scoort als framework voor virtual reality. Zo konden we een antwoord formuleren op alle onderzoeksvragen.



