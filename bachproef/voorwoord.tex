%%=============================================================================
%% Voorwoord
%%=============================================================================

\chapter*{Woord vooraf}
\label{ch:voorwoord}

%% TODO:
%% Het voorwoord is het enige deel van de bachelorproef waar je vanuit je
%% eigen standpunt (``ik-vorm'') mag schrijven. Je kan hier bv. motiveren
%% waarom jij het onderwerp wil bespreken.
%% Vergeet ook niet te bedanken wie je geholpen/gesteund/... heeft

Deze bachelorproef werd gemaakt om mijn opleiding toegepaste informatica met succes te kunnen voltooien. Het was voor mij niet gemakkelijk om een goed onderwerp te kunnen definiëren. Maar door de hulp van Karine Samyn, lector aan de hogent, heb ik toch een goed onderwerp kunnen vinden. Zij bracht mij namelijk op de hoogte van React VR. Een framework voor virtual reality applicaties te bouwen in de web browser. Aangezien ik al een gezonde interesse had voor javascript, was het onderwerp dus zeker iets voor mij. In mei 2018 werd het framework hernoemt naar React 360. Dit was vooral een naamswijziging maar had toch een kleine impact waardoor ik bepaalde delen van mijn bachelorproef moest aanpassen.

Bij het schrijven van deze bachelorproef heb ik hulp gekregen uit meerdere kanten.

Ik wil beginnen met mijn promotor Johan Van Schoor te bedanken voor de feedback op mijn bachelorproef. Daarnaast wil ik ook zeer graag mijn co-promotor Jasper Dansercoer bedanken. Hij heeft mij zeer veel bijgeleerd over het React framework en stond altijd klaar voor mijn vragen.

Uiteraard wil ik ook graag mijn goede vrienden Arne Aers en Maxim Hendrickx bedanken voor het nalezen van mijn bachelorproef en feedback hierop te geven.

Ten slotte wil ik graag nog de mede-studenten  Niels Blanckaert en Giel De Clercq bedanken waarmee ik 3 jaar lang de opleiding met plezier heb gevolgd en meerdere projecten met succes heb afgewerkt.