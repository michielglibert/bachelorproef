\chapter{Stand van zaken}
\label{ch:stand-van-zaken}

% Tip: Begin elk hoofdstuk met een paragraaf inleiding die beschrijft hoe
% dit hoofdstuk past binnen het geheel van de bachelorproef. Geef in het
% bijzonder aan wat de link is met het vorige en volgende hoofdstuk.

% Pas na deze inleidende paragraaf komt de eerste sectiehoofding.

%%=============================================================================
%% Virtual reality
%%=============================================================================
Voordat React 360 kan worden bekeken is het belangrijk een goed beeld te scheppen van virtual reality zelf. We gaan dieper in op wanneer iets virtual reality is, hoe het werkt en wat de gevolgen zijn voor de gebruiker van deze applicaties. We zullen tot slot, nog een overzicht geven van mogelijke alternatieven voor React 360 met ook een korte uitleg van deze frameworks.

\section{De virtuele realiteiten}
\label{sec:wanner-vr}
We weten al wat virtual reality is, maar wanneer is iets nu virtual reality. Er zijn namelijk nog een aantal andere vormen van valse realiteiten die al snel verward kunnen worden met virtual reality, maar wat zijn deze nu precies en wanneer kunnen we iets als VR aanzien.

\subsection{Virtual reality}
\label{subsec:virtual-reality}
\autocite{Sherman2000}
De 4 keys

\subsection{3D en 4D}
\label{subsec:3d-4d}
3D wil letterlijk zeggen driedimensionaal. Hiermee bedoelt men dat iets 3 meetkundige dimensies heeft namelijk diepte, breedte en hoogte \autocite{Wikipedia3D2018}. Deze technologie kan men dan gaan toepassen op beeldschermen a.d.h.v. stereoscopie, waar we in hoofdstuk \ref{sec:hoe-werkt-vr} wat meer uitleg over geven. Op deze manier krijgt de gebruiker dus een illusie van diepte. Meestal worden hier dan speciale brillen voor gebruikt ook wel 3D-brillen genoemd. De klassiekere brillen hebben dan een kleurfilter, meestal rood en blauw. Rood laat enkel rood door en blauw enkel blauw. Hiermee kan men dan vanaf 2 perspectieven een blauw afbeelding en een rode afbeelding gaan weergeven waardoor de diepte van een object dus zichtbaar is.

3D wordt al veel toegepast in de filmwereld, vooral in de cinema dan. Hierdoor ontstond dan ook de uitgebreidere 4D maar is eerder een marketingterm dan een technologie en wordt bijna uitsluitend in de filmwereld gebruikt als entertainment. Bij 4D gaat men de beelden nog altijd driedimensionaal gaan weergeven maar zorgt men ook voor extra fysieke effecten die synchroon lopen met de film. Als bijvoorbeeld een film zich in het water afspeelt kan men af en toe kleine spatjes water op de kijker schieten.

Het grote verschil hier is de immersion die amper aanwezig is. Je hebt wel een extra dimensie maar je kan niet rond je kijken, dat kan je met een virtual reality headset natuurlijk wel. Daarnaast is interactie met de wereld in de meeste gevallen niet aanwezig, spelconsoles zoals de Nintendo 3DS vormen dan weer een uitzondering. 3D of 4D is dus zeker geen volwaardige virtual reality, maar is wel al een stap in de goede richting.

\subsection{Augmented Reality}
\label{subsec:augmented-reality}
Augmented Reality wordt heel veel verward met virtual reality maar zijn toch 2 andere technologieën. Je kan augmented reality gaan vertalen naar toegevoegde realiteit. In tegenstelling tot virtual reality gaat men bij augmented reality letterlijk iets gaan toevoegen aan wat al bestaat. Bij virtual reality gaat men dan iets volledig virtueel gaan scheppen. Aangezien augemented reality iets toevoegt, is er ook niet verplicht nood aan een bril. Vandaag de dag wordt augmented reality regelmatig gebruikt bij de smartphone. Denk maar aan een applicatie waarbij je een object kan afbeelden op een plaats. Bijvoorbeeld een bepaald meubilair die virtueel gegeneerd en getoond wordt in een kamer dat echt bestaat.

\begin{figure}
	\centering
	\includegraphics[width=0.7\linewidth]{Milgram.jpg}
	\caption{Augmented reality tegenover virtual reality.}
	\label{fig:ar-vs-vr}
\end{figure}

Er is niet echt een definitie van wat augmented reality precies inhoud. Je kan eigenlijk al een scorebord bij live uitzending van een sport op de televisie zien als een vorm van augmented reality. Maar volgens \autocite{Azuma1997} zou iets aan volgende 3 karakteristieken moeten voldoen om augmented reality te kunnen zijn:

\begin{itemize}
	\item Combineert de echte wereld met de virtuele wereld
	\item Men kan interacties doen met het virtuele
	\item De virtuele wereld bezit de 3 dimensies (diepte, breedte en hoogt)
\end{itemize}

Als we deze 3 karakteristieken toepassen op het voorbeeld van het scorebord, zien we dus dat een scorebord bij een live uitzending van sport geen augmented reality is. De 1ste voorwaarde wordt wel voldaan, maar de 2de en de 3de niet. Men kan geen interacties gaan uitvoeren op dit scorebord, de programmamaker zou dit wel kunnen dus voor hem is dan enkel de 3de regel niet voldaan. Zou het beeld voor de programmamaker dan ook nog in 3D (een 3D scorebord dus, niet noodzakelijk een 3D beeldscherm) worden weergegeven, dan kunnen we stellen dat dit een vorm van augmented reality is.


\section{De werking van virtual reality}
\label{sec:hoe-werkt-vr}
Er werd al duidelijk gemaakt wat virtual reality precies is, maar het is ook belangrijk te weten hoe de technologie werkt. Er zal hier niet te technisch worden op ingegaan aangezien dit niet enorm relevant is voor deze bachelorproef.

\subsection{Stereoscopie}
\label{subsec:stereoscopie}
Bij virtuele werkelijkheid wordt er gebruik gemaakt van een illusie. Dit doet men ten eerste a.d.h.v. stereoscopie. Hierbij gaat men diepte meegeven aan een afbeelding. Dit doet men door 2 afbeeldingen vanaf een verschillende afstand (meestal de afstand tussen de ogen) te maken. Hierna gaat men dit combineren tot één stereoafbeelding. Deze technologie wordt ook gebruikt bij 3D. 

\begin{figure}
	\centering
	\includegraphics[width=0.7\linewidth]{stereoscopie.jpg}
	\caption{Stereoscopie voorbeeld.}
	\label{fig:stereoscopie}
\end{figure}

Door stereoscopie kan men dus diepte gaan simuleren. Hiermee is al een een grote voorwaarde voldaan om iets realistisch te laten lijken, namelijk 'Immersion' (key elements VR), onderdompeling. Zo kan men beter inschatten hoe ver en hoe groot een object is in de virtuele wereld. Men kan deze 2 afbeeldingen dan gaan tonen door zo een VR headset. Dit kan op meerdere manieren.
\begin{itemize}
	\item Ofwel maakt men gebruik van een smartphone waarbij men dan een headset heeft waarbij gebruik wordt gemaakt van lenzen. Op de smartphone worden er 2 beelden geprojecteerd en zorgen de lenzen ervoor dat de omgeving ruimer lijkt dan het werkelijk is. Hiermee kan men dan op een goedkope manier virtuele realiteit gaan tonen. Maar dit is meestal ten koste van de kwaliteit doordat de resolutie op smartphones meestal te laag is voor virtuele werkelijkheid scherp te kunnen weergeven.
	
	\item Daarnaast heeft men de meer geavanceerde virtuele headsets. Hierbij is er per oog een scherm met een hoge resolutie. Deze tonen dan elk hun beeld en simuleren dan de virtuele omgeving. Deze headsets worden dan ook het meest gebruikt bij gaming, maar zijn automatisch ook een pak duurder en vereisen een krachtige computer.
\end{itemize}

\subsection{De virtuele ervaring}
\label{subsec:vr-ervaring}
In hoofdstuk \ref{subsec:3d-4d} hebben we al vermeld dat het niet enkel door de extra dimensie is dat iets al virtual reality is. Er zijn nog meerdere aspecten waarmee men rekening moet houden bij een virtuele headset. Zo zijn de beelden per seconde, ofwel frames per second (FPS), van de beeldschermen in die headsets zeer belangrijk. De 2 bekendste virtuele headsets momenteel op de markt, die dan vooral naar gaming gericht zijn, zijn de HTC vive en de Oculus Rift. Beide headsets maken gebruik van een 90Hz scherm. Dat wil zeggen dat er maximaal 90 beelden per seconde kunnen worden weergegeven. Hoe sneller, hoe realistischer. Een andere bekende headset, de Playstation VR, draait maar op 60Hz, 60 beelden per seconde dus. Bij smartphones kan dit nog lager liggen, tot 30Hz zelfs. Het is dus duidelijk dat de HTC vive en Oculus Rift een realistische ervaring zullen geven dan de Playstation VR of de smartphone.

Wat ook en grote impact kan hebben op de ervaring is latency. Latency is de hoeveelheid tijd er tussen zit als je een bepaalde input geeft en die dan wordt weergegeven in de virtuele wereld. Een goed voorbeeld hiervan is het moment dat je een stap vooruit zet, zal er een bepaalde hoeveelheid tijd zijn tot je vooruit beweegt in de virtuele wereld. Bij een te hoge latency zal de virtuele ervaring als heel slecht ervaren worden en zelfs vanaf er al een latency is van 20ms zal het menselijk brein duidelijk onderscheidt kunnen maken tussen iets dat echt is en iets dat vals is. Dit zal niet alleen meiden tot een mindere ervaring voor de gebruiker maar kan ook tot fysieke pijn leiden zoals motion sickness, iets dat ook voorkomt bij onze moderne manieren van transport bij bijvoorbeeld autorijden (wagenziekte). Het is dus zeer belangrijk dat deze latency zo laag mogelijk is, niet alleen voor de ervaring maar ook om ongemakkelijkheid bij de gebruiker te vermijden.

Ten laatste is de FOV, field of view ofwel het gezichtsveld in de virtuele wereld ook zeer belangrijk. Een mens ziet ongeveer 180° rond zich, maar dit kan oplopen tot 270° als er met de ogen bewogen wordt. De meeste headsets hebben maar een gezichtsveld tussen de 90° en 110°, wat dus eigenlijk niet genoeg is. Dit heeft dan ook een impact op de virtuele ervaring en kan ook hier opnieuw leiden tot motion sickness.

Als er dus niet wordt voldaan aan een hoge FPS, voldoende FOV en een lage latency zal dit dus een impact hebben op de virtuele ervaring. Hierdoor ontstaan dan fysieke ongemakken omdat het menselijk brein weet dat er iets niet klop. Het is dus zeer belangrijk aan ontwikkelaars van virtuele applicaties om hiermee rekening te houden. Als een applicatie je ziek maakt, dan ga je dat automatisch ook minder of zelfs niet meer gebruiken.

\subsection{Interactie met de virtuele wereld}
\label{subsec:interactie-vr}
Vandaag zijn we in staat om een virtuele wereld zeer realistisch weer te geven, maar toch blijf interactie met de virtuele wereld een moeilijk aspect van virtual reality en het is nochtans één van de belangrijkste. Wat men al sowieso goed doet, is rond je kunnen kijken in de virtuele wereld, dit werd al in de introductie vermeldt en gebeurt aan de hand van een gyroscoop. Maar hoe kan men dan bewegingen van de echte wereld gaan omzetten in die virtuele wereld?

Dit kan ten eerste gebeuren aan de hand van afstandsbediening. Men kan gebruik maken van een simpele afstandbediening met joystick. Waarbij men dus gewoon in de virtuele wereld gaat bewegen als hoe men dit zou doen in een videospel, maar dit is natuurlijk niet de perfecte ervaring aangezien er weinig rekening wordt gehouden met hoe de handen bewegen. Een betere oplossing hiervoor waren dan de controllers die ook bewegingen registreren (denk maar aan de Nintendo Wii). Dit is al een grote stap vooruit. Verder gaan we hier niet dieper op ingaan.

\section{De gevolgen voor de gebruiker}
\label{sec:gevolgen-vr}
Anders dan een gewone app...

\subsection{User Experience}
\label{subsec:user-experience}
Hier dieper ingaan op hoe het moet gebouwd worden => piramide
\autocite{Cronin2015}

\begin{figure}
	\centering
	\includegraphics[width=0.7\linewidth]{MaslowCroninPiramide.png}
	\caption{De hiërarchie van de benodigdheden in virtual reality.}
	\label{fig:maslowcroninpiramide}
\end{figure}

\subsection{Fysiek}
\label{subsec:fysiek}
Hier dieper ingaan op de fysieke gevolgen (misselijkheid)
\autocite{Pappas2016}

\section{Javascript}
\label{sec:frameworks}
-> Hier focus leggen op wat is javascript

\subsection{Belangrijkste onderdelen}
\label{subsec:Web VR}
-> Enkele belangrijke onderdelen van javacsript uitleggen

\section{De mogelijke frameworks naast React 360}
\label{sec:frameworks}
Het ontwikkelen van virtual reality applicaties gaat natuurlijk niet zomaar en vereisen een goede technische kennis. Zeker het feit dat je in een wereld zit waar er volledig om zich kan worden gekeken en ook eventueel bewogen worden is natuurlijk een grote uitdaging. Om dit allemaal makkelijker te maken zijn er momenteel al een aantal frameworks beschikbaar die het ontwikkelen van virtual reality applicaties een pak makkelijker maken. Naast React 360 bestaan er nog een aantal mogelijke frameworks \autocite{UIUXLab2017}.

\subsection{Web VR}
\label{subsec:Web VR}

\subsection{Primrose}
\label{subsec:Primrose}

\subsection{Agron.js}
\label{subsec:Agron.js}

\subsection{A-Frame}
\label{subsec:A-Frame}²


