%%=============================================================================
%% Inleiding
%%=============================================================================

\chapter{Inleiding}
\label{ch:inleiding}
Virtual Reality, of in het Nederlands virtuele werkelijkheid, is een technologie die het mogelijk maakt om een bepaalde omgeving zowel auditief als visueel te simuleren en het gevoel te geven aan de gebruiker dat hij/zij zich werkelijk in die omgeving bevindt. Men gaat letterlijk op de menselijke zintuigen gaan inspelen door middel van elektronica om de gebruiker een gevoel van realiteit te geven die er eigenlijk niet is. Het voornaamste apparaat dat hiervoor gebruikt wordt is een virtual reality bril. Dit is een bril waarmee voor elk oog een beeld van de virtuele wereld wordt weergegeven. Deze brillen hebben meestal ook extra sensoren zoals bijvoorbeeld een gyroscoop die ook alle bewegingen met het hoofd opvolgt en weergeeft in de virtuele wereld.

De toepassingen waar dat virtual reality voor kan gebruikt worden zijn zeer uitgebreid. De bekendste hiervoor is entertainment, zoals gaming. Hiermee kan men simuleren alsof de persoon zich werkelijk in het spel bevindt. Daarnaast kan het ook gebruikt worden in de medische wereld. Men kan virtual reality of kort VR, gaan gebruiken als behandeling tegen bepaalde aandoeningen zoals fobieën en posttraumatische-stressstoornis. Vervolgens kan virtual reality gebruikt worden voor bepaalde ingrepen te gaan simuleren als training. Deze training kan dan een goede voorbereiding zijn op de echte ingreep. Het gebruik van VR als training komt ook voor in andere gebieden dan het medische, zoals in het leger, de ruimtevaart, vliegsimulators …

Virtual reality heeft een ruim aanbod van toepassingsgebieden waarvoor het gebruikt kan worden. Toch is het ontwikkelen van een virtual reality applicatie een grote uitdaging. Bijgevolg is er nood aan een goed framework die voldoende componenten aanbiedt voor de gewenste doeleinden van de VR-applicatie. React 360 is een door Facebook ontwikkeld framework dat gebruik maakt van React JS ( ook door Facebook) dat voor zowel 360\textdegree\ applicaties gebruikt kan worden als voor VR applicaties. Deze bachelorproef legt vooral de focus op de ontwikkeling van mobile virtual reality applicaties.

\section{Probleemstelling}
\label{sec:probleemstelling}

Virtual reality is een nog opkomende trend die nog in zijn kinderschoenen staat. De prijzen voor een geavanceerde VR headset kunnen soms oplopen tot 900 euro. Dit is momenteel nog een struikelblok voor de consument om de aankoop van een VR  headset uit te stellen, echter is dit niet de enigste. In het artikel van \autocite{Abarrera2017} wordt goed aangehaald wat precies de redenen zijn waarom VR headsets nog niet volledig aangeslaan zijn bij het publiek. Een ander zeer belangrijk probleem momenteel met VR is het tekort aan ontwikkelaars voor deze technologie. Een virtual reality applicatie is niet hetzelfde als een gewone webapplicatie. Ten eerste moet er letterlijk in 360\textdegree\  gedacht worden aangezien een persoon in VR perfect rond zich moet kunnen kijken. Daarnaast mogen we niet vergeten dat bepaalde beelden die ongepast zijn voor VR een negatieve weerslag kunnen geven aan de gebruiker en zelfs kan leiden tot fysieke pijn. Ten slotte gaat het ontwikkelen van een virtual reality applicatie niet zomaar. Er is al snel nood aan een uitgebreide kennis over het ontwikkelen van 3D omgevingen zoals in videospellen. Dit kan voor een ontwikkelaar geen gemakkelijke taak zijn om dit zomaar aan te leren en dit zorgt er dan weer voor dat ontwikkelaars moeilijker de stap naar het ontwikkelen van virtual reality applicaties nemen.

React 360 zou hier een oplossing voor moeten bieden. Dit framework biedt niet alleen een groot aantal componenten aan waarmee er al snel een solide VR applicatie kan worden ontwikkeld, maar daarnaast wordt er bij React 360 ook gebruik gemaakt van React JS. React JS is al een gekend framework voor het ontwikkelen van interactieve webapplicaties. Dit zorgt ervoor dat de kloof tussen het ontwikkelen van een webapplicatie en een virtual reality applicatie heel wat kleiner wordt. Ook is het zo dat men voor deze applicaties uit te voeren niet verplicht een virtual reality headset hoeft te gebruiken aangezien deze applicaties in 360\textdegree\ via een webbrowser kunnen uitgevoerd worden en het VR gedeelte ervan eerder een uitbreiding erop is. Daarnaast zijn de mobiele virtual headsets, waarop we de nadruk gaan leggen in dit onderzoek, veel goedkoper. Hier spreken we eerder over prijzen van 20 tot 50 euro. De vraag is echter of React 360 wel een goed framework is in zijn huidige vorm. Kunnen we er wel goede VR applicaties mee ontwikkelen die gebruiksvriendelijk zijn en daarnaast weten we nog niet precies voor welke doeleinden het framework zou kunnen gebruikt worden. Een computer moet veel meer berekeningen doen voor een 3D omgeving te kunnen creëren. Is er dan ook nood aan goede en performante hardware? 


\section{Onderzoeksvraag}
\label{sec:onderzoeksvraag}

Deze bachelorproef zoekt antwoorden op volgende onderzoeksvragen:
\begin{itemize}
  \item	Hoe ervaart een gebruiker een VR-applicatie ontwikkeld met React 360 ten opzichte van een gewone webapplicatie?
  \begin{itemize}
  \item	Hoe voelt de user experience aan?
  \item	Zijn er fysieke klachten aanwezig?
  \end{itemize}
  \item	Wat is de performantie van React 360 in de browser?
\begin{itemize}
	\item Is er nood aan dure hardware voor React 360?
\end{itemize}
  \item	Voor welke doeleinden kan React 360 gebruikt worden?
  \item	Is React 360 al een goed framework voor volwaardige VR-applicaties?
  \item Is React 360 voor ontwikkelaars makkelijk aan te leren?
\end{itemize}


\section{Onderzoeksdoelstelling}
\label{sec:onderzoeksdoelstelling}

In dit onderzoek zochten we een antwoord op al de onderzoeksvragen en vooral of React 360 een goed framework is voor gebruikersvriendelijke en solide virtual reality applicaties te ontwikkelen en voor welke doeleinden. Het antwoord hoefde niet noodzakelijk positief te zijn aangezien het belangrijkste was om een duidelijk antwoord te formuleren. We wouden vooral het framework gaan ontleden en alle belangrijkste aspecten dat het aanbiedt bekijken terwijl we ook rekening hielden met alle concepten van de wereld van virtual reality. We hielden ook in gedachten dat virtual reality een nieuwe technologie is. Het concept hierachter bestond al lang op het moment van het onderzoek, maar het was pas sinds de laatste jaren dat het zijn intrede had gemaakt op de markt.

\section{Opzet van deze bachelorproef}
\label{sec:opzet-bachelorproef}

% Het is gebruikelijk aan het einde van de inleiding een overzicht te
% geven van de opbouw van de rest van de tekst. Deze sectie bevat al een aanzet
% die je kan aanvullen/aanpassen in functie van je eigen tekst.

De rest van deze bachelorproef is als volgt opgebouwd:

In hoofdstuk~\ref{ch:stand-van-zaken} wordt een overzicht gegeven van de stand van zaken binnen het onderzoeksdomein, op basis van een literatuurstudie. Onder andere, de technologie virtual reality zelf en de programmeertalen waarmee React 360 gebouwd is: HTML, CSS en Javascript.

In hoofdstuk~\ref{ch:methodologie} wordt de methodologie toegelicht en worden de gebruikte onderzoekstechnieken besproken om een antwoord te kunnen formuleren op de onderzoeksvragen.

In hoofdstuk~\ref{ch:react-360} wordt React 360 bestudeert en bekijken we enkele belangrijke onderwerpen van het framework.

In hoofdstuk~\ref{ch:react-360-apps} worden de resultaten van de bevraging gegeven. Dit onderzoek bestond uit twee applicaties die getest werden door een 15-tal personen. Daarnaast wordt ook kort weergegeven hoe deze applicaties presteerden.

In hoofdstuk~\ref{ch:conclusie} wordt tenslotte de conclusie gegeven en een antwoord geformuleerd op de onderzoeksvragen. Daarbij wordt ook een aanzet gegeven voor mogelijke toekomstige onderzoeken binnen dit domein.

