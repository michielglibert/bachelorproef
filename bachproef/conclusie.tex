%%=============================================================================
%% Conclusie
%%=============================================================================

\chapter{Conclusie}
\label{ch:conclusie}

%% TODO: Trek een duidelijke conclusie, in de vorm van een antwoord op de
%% onderzoeksvra(a)g(en). Wat was jouw bijdrage aan het onderzoeksdomein en
%% hoe biedt dit meerwaarde aan het vakgebied/doelgroep? Reflecteer kritisch
%% over het resultaat. Had je deze uitkomst verwacht? Zijn er zaken die nog
%% niet duidelijk zijn? Heeft het onderzoek geleid tot nieuwe vragen die
%% uitnodigen tot verder onderzoek?

Het React 360 framework is een zeer goede aanzet tot de ontwikkeling van virtual reality. Het framework heeft een lage instapdrempel in de zin van dat de ontwikkelaar niet noodzakelijk uitgebreide kennis nodig heeft om een virtual reality applicatie ermee te kunnen ontwikkelen. React JS is een framework dat zeer snel kan worden aangeleerd waardoor het ontwikkelen van deze applicaties ook automatisch een pak eenvoudiger wordt. Een ingewikkelde applicatie ontwikkelen wordt door middel van het React 360 framework een pak makkelijker gemaakt. Men kan zeer gebruiksvriendelijk interfaces gaan ontwikkelen om hiermee uitgebreide applicaties te gaan maken. Maar dit is waar het er momenteel bij React 360 bij blijft. De doeleinden voor het framework blijven zeer beperkt. We kunnen React 360 vooral gaan gebruiken voor applicaties waarbij nood is aan een goede user interface. Denk dus maar aan applicaties voor de toeristische sector waarbij men een rondleiding kan geven op bepaalde plaatsen met behulp van VR. Ook de medische sector kan dit eventueel gaan gebruiken, maar dit blijft ook zeer beperkt. Hierbij zou de focus dan eerder liggen op educatieve applicaties waarbij men een 3D weergave kan gaan geven van bepaalde organen. Ook immobiliën kantoren kunnen zeker gebruik maken van React 360 waarbij men 360\textdegree foto's of video's van huizen kan gaan voorstellen. De doeleinden van de applicaties doe ontwikkeld kunnen worden met React 360 blijven dus vooral 'professioneel' gericht. Voor entertainment kan men natuurlijk wel React 360 gebruiken, maar ook hier zeer beperkt. Simpele games zijn mogelijk, maar verre van wat er al mogelijk is met andere technologieën en VR headsets (HTC Vive, Oculus Rift, ..). 

De user experience was iets waar ik aan twijfelde, aangezien je 2D elementen in een 3D wereld gaat gaan weergeven. Dit bleek uiteindelijk, zeer goed te werken. Dit is ook de manier waarop gebruikers het duidelijk graag hebben volgens de resultaten van het onderzoek. Dit is dan zowel voor de ontwikkelaar als voor de gebruiker een groot voordeel. Het blijf natuurlijk nog steeds virtual reality. De fysieke opmerkingen waren dus ook regelmatig aanwezig. Vooral personen die last hadden van de ogen. Maar misselijkheid en duizeligheid kwam ook een aantal keer voor. Dit blijft dus nog steeds een belangrijk werkpunt voor de ontwikkelaar om hiermee rekening te houden.

Bij React 360 is er een actieve community aanwezig, iets wat dan weer een groot voordeel is voor de ontwikkelaar zelf. Andere frameworks zoals Primrose heeft geen actieve community en daardoor zal de ontwikkeling met dat framework ook een pak moeilijker verlopen. Zo bezit React 360 wel nog een groot aantal bugs. Dit was vooral dan te merken bij het cross platform gebruik. Zo verliep het uitvoeren van de testapplicaties op een desktop computer met de browser google chrome vlekkeloos. Alle functionaliteit werkte naar behoren. Bij de smartphone versie van chrome werden er dan weer enkele problemen ondervonden zoals het geluid dat niet correct afspeelde. Hiervoor werd dan ook een andere variant op chrome gebruikt, chrome canary, om dit te gaan oplossen, maar dit toont dan weer aan dat React 360 nog niet 100\% cross-platform is. Iets wat facebook nochtans aanhaalt op de homepage van het framework. Wat mij dan ook opviel is de beperkte documentatie die momenteel aanwezig is op de website. Zo werd in mei 2018 het framework aangepast van React VR naar React 360. De core functionaliteiten bleven hetzelfde, maar het hele framework kreeg een nieuwe look and feel. Zo is er een hoop functionaliteit die nog van React VR afkomstig is en dus ook nog steeds werkt in React 360, maar die nog steeds ongedocumenteerd is. Maar, react 360 doet het wel goed als hij het doet. Ookal was het batterij en geheugen verbruik wel hoog, ik ondervond geen onderbrekingen tijdens het uitvoeren van de applicaties. Daarbij is er niet noodzakelijk nood aan een zeer dure smartphone. Tegenwoordig is de prijs van een smartphone met 2GB aan RAM geheugen vrij laag.

Al bij al, kunnen we stellen dat React 360 zeker een goede stap in de juiste richting is. Het zet aan tot een verdere ontwikkeling en biedt ook een manier aan om op een gemakkelijke manier applicaties te ontwikkelen die wel degelijk gebruiksvriendelijk zijn, maar het aantal doeleinden blijft beperkt en is dus enkel bedoelt voor eenvoudige interactieve applicaties. Echte volwaardige applicaties? Zolang de applicatie eenvoudig blijft, dan wel. Willen we een ingewikkeldere applicatie, met gebruik van veel 3D elementen en animaties, dan is React 360 niet het juiste framework en zal de ontwikkelaar gebruik moeten maken van moeilijkere ontwikkelingsmethoden. React 360 blijft ten slotte wel een framework voor de browser en een browser heeft ook zo zijn beperkingen.