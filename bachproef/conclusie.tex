%%=============================================================================
%% Conclusie
%%=============================================================================

\chapter{Conclusie}
\label{ch:conclusie}

%% TODO: Trek een duidelijke conclusie, in de vorm van een antwoord op de
%% onderzoeksvra(a)g(en). Wat was jouw bijdrage aan het onderzoeksdomein en
%% hoe biedt dit meerwaarde aan het vakgebied/doelgroep? Reflecteer kritisch
%% over het resultaat. Had je deze uitkomst verwacht? Zijn er zaken die nog
%% niet duidelijk zijn? Heeft het onderzoek geleid tot nieuwe vragen die
%% uitnodigen tot verder onderzoek?

Het React 360 framework is een zeer goede aanzet tot de verdere ontwikkeling van virtual reality. Het framework heeft een lage instapdrempel in de zin van dat de ontwikkelaar niet noodzakelijk uitgebreide kennis nodig heeft om een virtual reality applicatie ermee te kunnen ontwikkelen. React JS is een framework dat zeer snel kan worden aangeleerd waardoor het ontwikkelen van deze applicaties ook automatisch heel wat eenvoudiger wordt. Een ingewikkelde applicatie ontwikkelen wordt door middel van het React 360 framework heel wat eenvoudiger gemaakt. Men kan zeer gebruiksvriendelijke interfaces gaan ontwikkelen om hiermee solide applicaties te gaan maken. Dit is echter waar het er momenteel bij React 360 bij blijft. De doeleinden en functionaliteit van applicaties gebouwd met het framework blijven zeer beperkt. We kunnen React 360 vooral gaan gebruiken voor applicaties waarbij nood is aan een goede user interface en die vooral professioneel gericht zijn. Bijvoorbeeld applicaties voor de toeristische sector waarbij men een rondleiding kan geven op bepaalde plaatsen met behulp van VR. Ook immobiliënkantoren kunnen gebruik maken van React 360 waarbij men bijvoorbeeld 360\textdegree\ foto's of video's van huizen kan gaan voorstellen. Zelfs de medische sector of scholen in het algemeen kunnen educatieve applicaties ontwikkelen met React 360 waarbij men een 3D weergave kan gaan geven van bepaalde organen. Voor  entertainment zoals een film in 360\textdegree\ bekijken, kan men wel React 360 gebruiken, maar veel verder dan dat gaat het niet op vlak van entertainment. Eenvoudige games zijn mogelijk, echter komt React 360 niet in de buurt van wat er al mogelijk is met de andere technologieën en VR headsets (HTC Vive, Oculus Rift, ..) op vlak van gaming. 

De user experience was iets waar ik initieel aan twijfelde aangezien er 2D elementen in een 3D wereld worden weergeven. Dit bleek uiteindelijk zeer goed samen te werken. Dit is ook de manier waarop gebruikers het duidelijk graag hebben volgens de resultaten van het onderzoek. Dit is zowel voor de ontwikkelaar als voor de gebruiker een groot voordeel. Het blijft natuurlijk nog steeds virtual reality. De fysieke opmerkingen werden ook dikwijls duidelijk tijdens het onderzoek. Dit waren regelmatig gebruikers die last hadden van de ogen, maar ook misselijkheid en duizeligheid kwamen een aantal keer voor. Fysieke klachten vermijden bij een VR applicatie is dus een belangrijk werkpunt voor de ontwikkelaar.

Bij React 360 is er een actieve community aanwezig. Andere frameworks zoals Primrose hebben geen actieve community en daardoor zal de applicatie ontwikkeling met dat framework ook een pak stroever verlopen door de weinige hulp die aanwezig is bij eventuele problemen. React 360 bezit wel nog een aantal bugs, maar dat is misschien eerder te wijten aan WebVR dan React 360 zelf. Dit was vooral te merken bij het cross-platform gebruik. Zo verliep het uitvoeren van de testapplicaties op een desktop computer met de browser 'Google Chrome' vlekkeloos. Bij de smartphone versie van Chrome werden er dan weer enkele problemen ondervonden zoals het geluid dat niet correct afspeelde. Om deze problemen te kunnen oplossen werd een variant op Chrome gebruikt, namelijk Chrome Canary. Dit toont dan weer aan dat React 360 nog niet 100\% cross-platform is. Iets wat Facebook wel aanhaalt op de homepage van het React 360 framework. Wat mij ook duidelijk werd in tegenstelling tot wat ik initieel dacht bij het aanvangen van mijn bachelorproef, is de beperkte documentatie die momenteel aanwezig is op de website. Zo werd in mei 2018 het framework aangepast van React VR naar React 360. De core functionaliteiten bleven hetzelfde en het hele framework kreeg een nieuwe 'look and feel'. Er is een groot aantal functionaliteit die nog van React VR afkomstig is en bijgevolg ook nog steeds werkt in React 360, echter zijn deze nog steeds ongedocumenteerd. React 360 doet het wel goed als hij het doet. Ook al was het batterij- en geheugen verbruik wel hoog. Ik ondervond geen onderbrekingen tijdens het uitvoeren van de applicaties. Daardoor is er niet noodzakelijk nood aan een zeer dure smartphone. Tegenwoordig is de prijs van een smartphone met 2GB RAM-geheugen relatief laag, maar er moet wel een zekere 'kracht' aanwezig zijn.

Al bij al kunnen we stellen dat React 360 zeker een stap in de juiste richting is. Het zet aan tot een verdere ontwikkeling en biedt ook een manier aan om applicaties te ontwikkelen die wel gebruiksvriendelijk zijn. Toch blijft het aantal doeleinden eerder beperkt en is dit enkel bedoelt voor eenvoudige en interactieve applicaties. Willen we een ingewikkeldere applicatie met gebruik van veel 3D elementen en animaties, dan is React 360 niet het ideale framework en zou een beter alternatief het A-Frame framework van Mozilla kunnen zijn. Mijn initiële verwachtingen voor React 360 lagen hoger dan het resultaat van deze bachelorproef aanhaalt.