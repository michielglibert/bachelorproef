%%=============================================================================
%% Samenvatting
%%=============================================================================

% TODO: De "abstract" of samenvatting is een kernachtige (~ 1 blz. voor een
% thesis) synthese van het document.
%
% Deze aspecten moeten zeker aan bod komen:
% - Context: waarom is dit werk belangrijk?
% - Nood: waarom moest dit onderzocht worden?
% - Taak: wat heb je precies gedaan?
% - Object: wat staat in dit document geschreven?
% - Resultaat: wat was het resultaat?
% - Conclusie: wat is/zijn de belangrijkste conclusie(s)?
% - Perspectief: blijven er nog vragen open die in de toekomst nog kunnen
%    onderzocht worden? Wat is een mogelijk vervolg voor jouw onderzoek?
%
% LET OP! Een samenvatting is GEEN voorwoord!

%%---------- Nederlandse samenvatting -----------------------------------------
%
% TODO: Als je je bachelorproef in het Engels schrijft, moet je eerst een
% Nederlandse samenvatting invoegen. Haal daarvoor onderstaande code uit
% commentaar.
% Wie zijn bachelorproef in het Nederlands schrijft, kan dit negeren, de inhoud
% wordt niet in het document ingevoegd.

\IfLanguageName{english}{%
\selectlanguage{dutch}
\chapter*{Samenvatting}
\lipsum[1-4]
\selectlanguage{english}
}{}

%%---------- Samenvatting -----------------------------------------------------
% De samenvatting in de hoofdtaal van het document

\chapter*{\IfLanguageName{dutch}{Samenvatting}{Abstract}}

React 360 is een virtual reality (VR) framework ontwikkeld door facebook dat gebasseerd is op het al bestaande en frequent gebruikte React JS framework dat tevens ook ontwikkeld is door facebook. Het concept achter virtual reality bestaat al lang maar heeft pas sinds zijn laatste jaren zijn intrede gemaakt op de consumentenmarkt. Aangezien een virtual reality wereld zich afspeelt in een wereld waar de gebruiker drie dimensionaal en in 360° rond zich kan kijken is het een grote uitdaging voor een ontwikkelaar om hiervoor applicaties te gaan ontwikkelen. Het React 360 framework zou hier de oplossing voor moeten bieden en geeft aan de ontwikkelaar de mogelijkheid om eenvoudige maar aangename virtual reality applicaties te creëren.

In deze bachelorproef onderzoeken we of React 360 een goed framework is voor een volwaardige VR applicatie. We bekijken voor welke doeleinden we het zouden kunnen gebruiken, wat de performantie is, hoe een gebruiker een applicatie ervaart gemaakt met dit framework en of het framework gemakkelijk aan te leren is voor een ontwikkelaar.

We gaan in eerste instantie een stand van zaken opstellen over de concepten van virtual reality. Het is belangrijk een goed beeld te hebben van wat deze technologie precies inhoud voor we dieper op React 360 kunnen ingaan. Een VR applicatie verschilt namelijk veel van een gewone web applicatie. We bekijken in deze stand van zaken ook de programmeertalen waarmee het framework is opgebouwd. Tot slot bekijken we enkele mogelijke alternatieven voor React 360.

Hierna beginnen we aan het echte onderzoek. In het 1ste deel van het onderzoek bekijken we kort de belangrijkste componenten van het React JS framework. React 360 is hier op gebasseerd en dus is dit essentieel om het te kunnen verstaan. Hierna gaan we het React 360 framework gaan ontleden. We overlopen alle concepten en geven hier een uitgebreide uitleg over.

Nadat we het framework kunnen begrijpen, gaan we over tot het 2de deel van het onderzoek. Hiervoor werden 2 applicaties ontwikkeld en getest bij 15 personen tussen de 20 en 25 jaar. Elke persoon kreeg 5-10 minuten om beide applicaties uit te testen met een virtual reality headset in combinatie met de smartphone. Tijdens het uitvoeren van dit onderzoek bekeken we ook hoe deze applicaties presteerden op de smartphones en of ze veel van de capaciteiten van de smartphone vroegen. Tot slot geven we de resultaten weer van dit onderzoek en maken we hier een besluit van op.

Uiteindelijk kunnen we dan een conclusie opmaken die een antwoord formuleert op al onze onderzoeksvragen.
